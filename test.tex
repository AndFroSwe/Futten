
\documentclass{article}

\usepackage[T1]{fontenc}
\usepackage[swedish]{babel}
\usepackage{listings}
\usepackage{color} %red, green, blue, yellow, cyan, magenta, black, white
\usepackage{amsmath}
\definecolor{mygreen}{RGB}{28,172,0} % color values Red, Green, Blue
\definecolor{mylilas}{RGB}{170,55,241}


\begin{document}


\lstset{language=Matlab,%
    %basicstyle=\color{red},
    breaklines=true,%
    morekeywords={matlab2tikz},
    keywordstyle=\color{blue},%
    morekeywords=[2]{1}, keywordstyle=[2]{\color{black}},
    identifierstyle=\color{black},%
    stringstyle=\color{mylilas},
    commentstyle=\color{mygreen},%
    showstringspaces=false,%without this there will be a symbol in the places where there is a space
    numbers=left,%
    numberstyle={\tiny \color{black}},% size of the numbers
    numbersep=9pt, % this defines how far the numbers are from the text
    emph=[1]{for,end,break},emphstyle=[1]\color{red}, %some words to emphasise
    %emph=[2]{word1,word2}, emphstyle=[2]{style},    
} 
\section*{Bilaga i: Huvudprogram}
Denna bilaga inneh�ller huvudkoden som anv�nts i uppgiften. 
\lstinputlisting{Futten_main.m} 
\newpage

\section*{Bilaga ii: Runge-Kutta 4}
Denna bilaga inneh�ller koden till Runge-Kutta 4 som anv�nts, samt dess
hj�lpfunktioner. \newline
\noindent\makebox[\linewidth]{\rule{\paperwidth}{0.4pt}}
RKeval itererar RK4 tills jorden passerats, allts� n�r $\dot{r}=0$.
\lstinputlisting{RKeval.m} 
\noindent\makebox[\linewidth]{\rule{\paperwidth}{0.4pt}}
RKstep utf�r ett steg av RK4.
\lstinputlisting{RK4step.m} 
\noindent\makebox[\linewidth]{\rule{\paperwidth}{0.4pt}}
F inneh�ller differentialekvationerna.
\lstinputlisting{F.m} 
\newpage

\section*{Bilaga iii: Passering}
Denna bilaga inneh�ller funktionen som anv�nts f�r att ber�kna Futtens
passeringsv�rden, samt Hermite- och linj�r interpolering.\newline
\noindent\makebox[\linewidth]{\rule{\paperwidth}{0.4pt}}
futten\_pass ber�knar v�rdena f�r futtens passering.
\lstinputlisting{futten_pass.m} 
\noindent\makebox[\linewidth]{\rule{\paperwidth}{0.4pt}}
linpol �r en funktion f�r linj�rinterpolering.
\lstinputlisting{linpol.m}
\noindent\makebox[\linewidth]{\rule{\paperwidth}{0.4pt}}
herm �r en funtion f�r Hermite-interpolering.
\lstinputlisting{herm.m}
\noindent\makebox[\linewidth]{\rule{\paperwidth}{0.4pt}}
bisection\_meth inneh�ller sekantmetoden.
\lstinputlisting{bisection_meth.m}
\newpage

\section*{Bilaga iv: �vriga funktioner}
Denna bilaga inneh�ller funktioner som inte lika l�tt kan presenteras i
sammanhang av uppgiftsl�sningen. \newline
\noindent\makebox[\linewidth]{\rule{\paperwidth}{0.4pt}}
least\_square �r en  funtion f�r minstakvadratanpassning.
\lstinputlisting{least_square.m}
\noindent\makebox[\linewidth]{\rule{\paperwidth}{0.4pt}}
cartesian g�r om pol�ra koordinater till kartesiska.
\lstinputlisting{cartesian.m}
\noindent\makebox[\linewidth]{\rule{\paperwidth}{0.4pt}}
arclength ber�knar banl�ngden f�r en kurva med givna punkter i kartesiska
koordinater.
\lstinputlisting{arclength.m}
\noindent\makebox[\linewidth]{\rule{\paperwidth}{0.4pt}}
Hermitefel r�knar ut felskattningen f�r Hermiteinterpolering.
\lstinputlisting{Hermitefel.m}
\noindent\makebox[\linewidth]{\rule{\paperwidth}{0.4pt}}
herm\_step Hermiteinterpolerar mellan tv� punkter och returnerar interpolerade
v�rden mellan kurvorna (anv�nds for att plotta hel interpolerad kurva).
\lstinputlisting{herm_step.m}


































\end{document}
